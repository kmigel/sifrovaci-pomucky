\documentclass{article}
\usepackage{hyperref}
\hypersetup{
    colorlinks=true,
    linkcolor=blue,
    filecolor=magenta,      
    urlcolor=cyan,
    pdfpagemode=FullScreen,
    }
\pagenumbering{gobble}

\title{Projekt - Šifrovací pomůcky}
\author{Anna-Kristina Migel}
\date{08/10/2024}

\begin{document}
\maketitle{}

\section{Co dělám}

Šifrovací pomůcky se využívají během šifer, aby si lidi nemuseli pokaždé vyhledávat morseovku, Braillovo písmo nebo třeba semafor. Tyto pomůcky jsou ale většinou papírové a je zdlouhavé si v tabulkách vyhledávat, jaké písmeno označuje nějaký semafor.

Mým projektem bude webová aplikace s šifrovacími pomůckami. Místo toho, aby lidi ručně hledali, co znamená "čárka tečka tečka", tak to stačí zadat do aplikace, a ta jim hned ukáže, že se jedná o písmeno D.

\section{Proč to dělám}

Samozřejmě už někoho přede mnou napadlo, že papírové pomůcky nejsou optimální, a tak byla vytvořena \href{https://play.google.com/store/apps/details?id=cz.absolutno.sifry&hl=cs}{tato aplikace}. Tato aplikace je ale pouze na android a já znám lidi, kteří by rádi tuto aplikaci používali, ale mají iPhone. Já mezi ně také patřím, proto jsem se rozhodla, že udělám webovou aplikaci, tím pádem ji můžou používat všichni nezávisle na operačním systému.

\section{Kdo to bude využívat}

Jak již bylo zmíněno výše, tak ji budou využívat lidi, kteří se účastní šifrovaček a chtěli by mít šifrovací pomůcky v mobilu, ale mají iPhone. V neposlední řadě tuto aplikaci plánuji využívat i já.

\section{Jak to budu dělat}

Moc zkušeností s tvorbou webových aplikací nemám, ale zajímá mě to a chci se to lépe naučit. Rozhodně budu používat Javascript a podívám se na jeho frameworky, vyzkouším si práci s některými z nich a jeden si vyberu.

\end{document}
