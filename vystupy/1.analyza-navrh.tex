\documentclass{article}
\usepackage{hyperref}
\hypersetup{
    colorlinks=true,
    linkcolor=blue,
    filecolor=magenta,      
    urlcolor=cyan,
    pdfpagemode=FullScreen,
    }
\pagenumbering{gobble}

\title{Projekt - Analýza a návrh}
\author{Anna-Kristina Migel}
\date{21/10/2024}

\begin{document}
\maketitle{}

\section{Analýza}

\subsection{Funkční požadavky}

\begin{itemize}
    \item šifrovací pomůcky
        \begin{itemize}
            \item morseovka
            \item Braille
            \item semafor
            \item převod binárky na písmena
        \end{itemize}
   \item zadávání inputu různými způsoby
       \begin{itemize}
            \item čísla
            \item čárky a tečky
            \item hodinové ručičky
        \end{itemize}
    \item vymazání celého inputu najednou
    \item změnit jednu hodnotu inputu bez mazání všech
    \item seřadit pomůcky podle sebe
\end{itemize}

\subsection{Nefunkční požadavky}

\begin{itemize}
    \item zobrazení outputu do 1 sekundy
    \item cachování inputu pro případ, že by uživatel omylem opustil konkrétní pomůcku
\end{itemize}

\subsection{Cílová skupina}

\begin{itemize}
    \item lidé řešící šifrovačky
        \begin{itemize}
            \item mají zkušenosti s papírovými pomůcky
            \item většinou dost inteligentní na použití webové aplikace
            \item očekávají všechny pomůcky jako na papíře
            \item očekávají jednoduchý UI na rychlé zadávaní inputu
        \end{itemize}
\end{itemize}

\subsection{Technické předpoklady}

\begin{itemize}
    \item webová aplikace
    \item responsivní design
    \item HTML, CSS, Javascript, frameworky (nejspíše React)
    \item připojení k internetu na načtení stránky
        \begin{itemize}
            \item možná nativní aplikace bez připojení
        \end{itemize}
\end{itemize}

\subsection{Rizika a možné problémy}

\begin{itemize}
    \item kompatibilita se všemi prohlížeči
    \item chyba v pomůckách
\end{itemize}

\newpage

\section{Návrh}

\subsection{Rozdělení na části}

\begin{itemize}
    \item frontend
        \begin{itemize}
            \item možnost výběru pomůcky
            \item jednoduché zadávání inputu
        \end{itemize}
    \item backend není potřeba
\end{itemize}

\subsection{Data}

\begin{itemize}
    \item ukládáme si input
    \item libovolný formát, .txt
\end{itemize}

\subsection{Technologie}

\begin{itemize}
    \item HTML
    \item CSS, nejspíše SASS
    \item Javascript
    \item ideálně framework, ze kterého pak jde udělat nativní mobilní aplikace
\end{itemize}

\end{document}